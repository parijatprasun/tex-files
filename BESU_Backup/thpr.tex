\documentclass[9pt]{beamer}
\mode<presentation>{
\usetheme{Rochester}
%\usetheme{Luebeck}
%\useinnertheme{rectangles}
%\useoutertheme{shadow}
%\usecolortheme{sidebartab}
%\usecolortheme{seahorse}
%\usecolortheme{lily}
\usecolortheme{rose}
%\usecolortheme{orchid}
}

%\setbeamercolor{frametitle}{bg=red!80!black}

\title[]{Application of Wavelet Transform in Condition Monitoring of Induction Motor}
\author[]{Parijat Prasun Pal\\{\footnotesize \emph{Student, M.E. -- Final Year}}}

%\author{\alert{Under the Guidance of}\\
%Dr. G. Bandyopadhyay\\
%\alert{and}\\
%Dr. P. Chattopadhyay\\
%}

%\titlegraphic{\includegraphics[width=12mm]{besulogo}}
\institute{
	\begin{columns}
	\column{0.33\textwidth}
	\begin{flushright}
	\includegraphics[height=12mm]{besulogo}
	\end{flushright}
	\column{0.67\textwidth}
	\vspace*{5pt}\\
		Department of Electrical Engineering\\
		Bengal Engineering and Science University, Shibpur\\
		Howrah –- 711 103\\
		West Bengal\\
		India
	\end{columns}
}

\date{2006}

\begin{document}

\begin{frame}
\maketitle
\end{frame}

%%%%%%%%%% SLIDE 2
\begin{frame}
\frametitle{Outline}
\begin{itemize} 
\item Introduction
\item Different Monitoring Techniques
\item Condition Monitoring of Induction Motor
\item Different Approaches to Condition Monitoring
\item Objective and Scope of Work
\item Why Wavelets?
\item Wavelet Analysis
\item Proposed Scheme of Work
\item Fast Fourier Transform (FFT)
\item Signal Analysis with FFT
\item Discrete Wavelet Transform (DWT)
\item Signal Analysis with DWT
\item Summary and Conclusion
\item References
\end{itemize}
\end{frame}

%%%%%%%%% SLIDE 3
\begin{frame}
%\section{Introduction}
\frametitle{Introduction}

\begin{block}{What is Condition Monitoring?}
    \begin{itemize} 
    \item Predictive maintenance
    \item Early detection of a problem 
    \item Identification of the need for maintenance based on the condition of the monitored equipment by a comprehensive programme of data collection and analysis
    \item Allows maintenance to be performed in a planned and systematic manner before an equipment fails.
    \end{itemize}
\end{block}

\begin{block}{Advantages of Condition Monitoring}
    \begin{itemize}
    \item Minimisation of overall operational cost and plant shutdown time
    \item Reduction in the number of unexpected failure
    \item More reliable scheduling tool for routine preventive maintenance programme
    \end{itemize}
\end{block}
\end{frame}

%%%%%%%%%%% SLIDE 4
\begin{frame}
%\section{Different Monitoring Techniques}
\frametitle{Different Monitoring Techniques}
    \begin{block}{Mechanical Techniques}  
    \begin{itemize}
    \item Vibration Monitoring
    \item Speed Fluctuation
   \end{itemize} 
    \end{block}
    \begin{block}{Electrical Techniques} 
    \begin{itemize}
    \item Monitoring of electrical signals like voltage, current, instantaneous power, and their spectral components, axial flux, air-gap torque
    \item Stator current: \alert{Motor Current Signature Analysis (MCSA)}
    \end{itemize}
    \end{block}
    \begin{block}{Chemical Techniques} 
    Chemical degradation of various chemical substances
    \end{block}
    \begin{block}{Thermal Techniques} 
    Measurement of temperature
    \end{block}
\end{frame}

%%%%%%%%% SLIDE 5
\begin{frame}
%\section{Condition Monitoring of Induction Motor}
\frametitle{Condition Monitoring of Induction Motor}

\begin{block}{Faults and Monitoring Techniques}
    \begin{itemize}
    \item \alert{Stator Faults}: Stator current signature analysis
    \item \alert{Rotor Faults}: Stator current signature analysis, vibration and air-gap monitoring
    \item \alert{Bearing Faults}: Vibration and stator current monitoring method
    \item \alert{Air-Gap  Eccentricities}: Stator core vibration and the stator current monitoring
    \end{itemize}
\end{block}
\end{frame}

%%%%%%%%% SLIDE 6
\begin{frame}
%\section{Different Approaches to Condition Monitoring}
\frametitle{Different Approaches to Condition Monitoring}

\begin{block}{Different Approaches}
    \begin{itemize}
    \item Model-Based Approaches: Traditional lumped-parameter modeling and analysis of faulty motor performance
    \item Thermal Monitoring Approaches
    \item Signal-Processing Approaches
            \begin{itemize}
            \item Vibration or motor terminal current, voltage, or instantaneous power waveforms.
            \item Motor Current Signature Analysis: traditional Fourier transform techniques used for spectral analysis
            \end{itemize}
    \item Emerging Technology Approaches
            \begin{itemize}
            \item Neural networks, fuzzy logic and AI techniques
            \item Instantaneous power FFT, Park's transformation, bispectrum, high resolution spectral analysis
            \item \alert{Wavelets}: extensively used for signature analysis
            \end{itemize}
    \end{itemize}
\end{block}
\end{frame}

%%%%%%%%% SLIDE 7
\begin{frame}
%\section{Objective and Scope of Work}
\frametitle{Objective and Scope of Work}

\begin{block}{Objective}
    \begin{itemize}
    \item To design and develop an on-line monitoring and incipient fault detection scheme of induction motors by assessing the signature of the motor line current.
    \item This  project  will concentrate on induction motor faults due to broken rotor bar, bearing faults and eccentric rotor only applying  advanced  signal processing techniques like wavelet analysis.
    \end{itemize}
\end{block}

\begin{block}{Scope of Work}
    \begin{itemize}
    \item Wavelet transform was introduced with the idea of overcoming the difficulties with the traditional signal-processing tools like Fourier Transform and STFT.
    \item Proposed project work aims to eliminate all the limitations of present technology by using advanced signal processing tools of wavelet transform and related methods.
    \end{itemize}
\end{block}
\end{frame}

%%%%%%%%% SLIDE 8
\begin{frame}
%\section{Why Wavelets?}
\frametitle{Why Wavelets?}

\begin{columns}
\column{0.5\textwidth}
\begin{block}{Fourier Analysis}
    \begin{itemize}
    \item Very useful where signals are stationary 
    \item Generally used for induction motor fault detection
    \item \alert{Drawback}: Not appropriate for signals having a transitory characteristic
    \end{itemize}
\end{block}
\column{0.5\textwidth}
\begin{block}{Short-Time Fourier Transform (STFT)} 
    \begin{itemize}
    \item Windowing technique to analyse small sections of the signal at a time
    \item \alert{Drawback}: Fixed size of the window -– limited precision
    \end{itemize}
\end{block}
\end{columns}

\begin{block}{Wavelet Transform}
    \begin{itemize}
    \item A windowing technique with variable-size region
    \item More precise information on both low and high frequencies
    \item Helps perform stator current signal analysis during transients
    \item Can be used for a localised analysis in the time-frequency or time-scale domain
    \end{itemize}
\end{block}
\end{frame}

%%%%%%%%% SLIDE 9
\begin{frame}
%\section{Wavelet Analysis}
\frametitle{Wavelet Analysis}

\begin{columns}
\column{0.5\textwidth}
\begin{block}{Basic Theory}
    \begin{itemize}
    \item To adopt a wavelet prototype function, called an analysing wavelet or \alert{mother wavelet}
    \item Temporal analysis is performed with a contracted, high-frequency version of the prototype wavelet, 
    \item Frequency analysis is performed with a dilated, low-frequency version of the same wavelet.
    \item The original signal or function can be represented in terms of wavelet expansion (using coefficients in a linear combination of the wavelet functions).
    \end{itemize}
\end{block}
\column{0.5\textwidth}
\begin{block}{Some Wavelets}
\includegraphics[width=\textwidth]{mw}
(a) Haar, (b) Daubechies, (c) Morlet and (d) Mexican hat
\end{block}
\end{columns}
\end{frame}

\begin{frame}
\frametitle{Wavelet Analysis \hfill{2}}
\begin{block}{Wavelet Analysis Techniques}
    \begin{itemize}
    \item \alert{Continuous Wavelet Transform}: The sum over all time of the signal multiplied by scaled, shifted versions of the wavelet function generated from the so-called mother wavelet
    \item \alert{Discrete Wavelet Transform}: Mother wavelet is dilated and translated discretely
    \item \alert{Wavelet Packet Transform}: Particular linear combinations of wavelets
    \end{itemize}
\end{block}
\end{frame}

%%%%%%%%%%%%% SLIDE 10 
\begin{frame}
%\section{Proposed Scheme of Work}
\frametitle{Proposed Scheme of Work}

\begin{columns}
\column{0.5\textwidth}
\begin{picture}(300,180)(0,0)
\setlength{\unitlength}{0.5pt}
{\scriptsize
%\begin{picture}(300,320)(0,0)
\put(0,0){\framebox(120,40){\shortstack{Post-processing and\\ diagnosis}}}
\put(180,0){\framebox(120,40){\shortstack{Feature extraction\\ algorithm}}}
\put(50,80){\framebox(240,40){\shortstack{Signal procesing of the motor line\\ current using wavelet analysis}}}
\put(0,160){\framebox(120,60){\shortstack{A/D converter and\\ Data Acquisition\\ System}}}
\put(80,260){\framebox(100,60){\shortstack{Machinery\\ Fault\\ Simulator}}}
\put(240,280){\framebox(60,20){Load}}

\put(0,280){\vector(1,0){80}}
\put(0,290){\vector(1,0){80}}
\put(0,300){\vector(1,0){80}}

\put(20,280){\vector(0,-1){60}}
%\put(30,290){\vector(0,-1){70}}
%\put(40,300){\vector(0,-1){80}}

\put(180,290){\vector(1,0){60}}

\put(70,160){\vector(0,-1){40}}

\put(240,80){\vector(0,-1){40}}

\put(180,20){\vector(-1,0){60}}

\put(50,240){Line Currents}
%\end{picture}

}
\end{picture}
\column{0.5\textwidth}
\begin{block}{3-phase Induction Motor}
Machinery Fault Simulator (MFS)
\end{block}
\begin{block}{Signal Processing}
    \begin{itemize}
    \item Fast Fourier Transform (FFT)
    \item Discrete Wavelet Transform (DWT)
    \end{itemize}
\end{block}
\end{columns}
\end{frame}

%%%%%%%%%%%%% SLIDE 11
\begin{frame}
%\section{Fast Fourier Transform (FFT)}
\frametitle{Fast Fourier Transform (FFT)}

\begin{block}{Discrete Fourier Transform (DFT)}
    \begin{itemize}
    \item Formula: $\displaystyle{X(k) = \sum_{n=0}^{N-1}x(n)W_{N}^{kn}}$, where $W_{N} \stackrel{\triangle}{=} e^{-j2\pi/N}$
    \item $N^{2}$ floating-point operations, not economical for even moderate data length
    \end{itemize}
\end{block}
\begin{block}{Fast Fourier Transform (FFT)}
    \begin{itemize}
    \item Faster, approximately $N \log_{2} N$ operations
    \item \alert{Cooley-Tuckey} and other techniques: DFT of length $N$ ``decimated'' into successive smaller DFTs, two varieties -- \alert{decimation-in-time} and \alert{decimation-in-frequency}
    \end{itemize}
\end{block}
\end{frame}

%%%%%%%%%%%%%% SLIDE 12
\begin{frame}
\frametitle{Fast Fourier Transform (FFT) \hfill{2}}

\begin{block}{Decimation-in-frequency}
\begin{picture}(370,160)(0,0)
\setlength{\unitlength}{0.6pt}
%========== 8th line
\put(0,5){\small{$x(7)$}}
\put(0,0){\vector(1,0){28}}
\put(30,0){\circle*{5}}
\put(30,0){\vector(1,0){68}}
\put(100,0){\circle*{5}}
\put(100,5){\small{+}}
\put(100,0){\vector(1,0){28}}
\put(101,-8){--}
\put(110,5){\small{$W_{8}^{3}$}}
\put(130,0){\circle*{5}}
\put(130,0){\vector(1,0){68}}
\put(200,5){\small{+}}
\put(200,0){\circle*{5}}
\put(201,-8){--}
\put(210,5){\small{$W_{8}^{2}$}}
\put(200,0){\vector(1,0){28}}
\put(230,0){\circle*{5}}
\put(230,0){\vector(1,0){68}}
\put(300,5){\small{+}}
\put(300,0){\circle*{5}}
\put(301,-8){--}
\put(310,5){\small{$W_{8}^{0}$}}
\put(300,0){\vector(1,0){28}}
\put(330,0){\circle*{5}}
\put(330,0){\vector(1,0){38}}
\put(340,5){\small{$X(7)$}}
%========== 7th line
\put(0,40){\small{$x(6)$}}
\put(0,35){\vector(1,0){28}}
\put(30,35){\circle*{5}}
\put(30,35){\vector(1,0){68}}
\put(100,35){\circle*{5}}
\put(100,40){\small{+}}
\put(100,35){\vector(1,0){28}}
\put(101,27){--}
\put(110,40){\small{$W_{8}^{2}$}}
\put(130,35){\circle*{5}}
\put(130,35){\vector(1,0){68}}
\put(200,40){\small{+}}
\put(200,35){\circle*{5}}
\put(201,27){--}
\put(210,40){\small{$W_{8}^{0}$}}
\put(200,35){\vector(1,0){28}}
\put(230,35){\circle*{5}}
\put(230,35){\vector(1,0){68}}
\put(300,40){\small{+}}
\put(300,35){\circle*{5}}
\put(300,27){\small{+}}
%\put(310,25){\small{$W_{8}^{0}$}}
\put(300,35){\vector(1,0){28}}
\put(330,35){\circle*{5}}
\put(330,35){\vector(1,0){38}}
\put(340,40){\small{$X(3)$}}
\put(230,35){\vector(2,-1){68}}
\put(230,0){\vector(2,1){68}}
%========== 6th line
\put(0,75){\small{$x(5)$}}
\put(0,70){\vector(1,0){28}}
\put(30,70){\circle*{5}}
\put(30,70){\vector(1,0){68}}
\put(100,70){\circle*{5}}
\put(100,75){\small{+}}
\put(100,70){\vector(1,0){28}}
\put(101,62){--}
\put(110,75){\small{$W_{8}^{1}$}}
\put(130,70){\circle*{5}}
\put(130,70){\vector(1,0){68}}
\put(200,75){\small{+}}
\put(200,70){\circle*{5}}
\put(200,62){\small{+}}
%\put(210,75){\small{$W_{8}^{2}$}}
\put(200,70){\vector(1,0){28}}
\put(230,70){\circle*{5}}
\put(230,70){\vector(1,0){68}}
\put(300,75){\small{+}}
\put(300,70){\circle*{5}}
\put(301,62){--}
\put(310,75){\small{$W_{8}^{0}$}}
\put(300,70){\vector(1,0){28}}
\put(330,70){\circle*{5}}
\put(330,70){\vector(1,0){38}}
\put(340,75){\small{$X(5)$}}
%========== 5th line
\put(0,110){\small{$x(4)$}}
\put(0,105){\vector(1,0){28}}
\put(30,105){\circle*{5}}
\put(30,105){\vector(1,0){68}}
\put(100,105){\circle*{5}}
\put(100,110){\small{+}}
\put(100,105){\vector(1,0){28}}
\put(101,97){--}
\put(110,110){\small{$W_{8}^{0}$}}
\put(130,105){\circle*{5}}
\put(130,105){\vector(1,0){68}}
\put(200,110){\small{+}}
\put(200,105){\circle*{5}}
\put(200,97){\small{+}}
%\put(210,110){\small{$W_{8}^{0}$}}
\put(200,105){\vector(1,0){28}}
\put(230,105){\circle*{5}}
\put(230,105){\vector(1,0){68}}
\put(300,110){\small{+}}
\put(300,105){\circle*{5}}
\put(300,97){\small{+}}
%\put(310,25){\small{$W_{8}^{0}$}}
\put(300,105){\vector(1,0){28}}
\put(330,105){\circle*{5}}
\put(330,105){\vector(1,0){38}}
\put(340,110){\small{$X(1)$}}
\put(230,105){\vector(2,-1){68}}
\put(230,70){\vector(2,1){68}}

\put(130,105){\vector(1,-1){68}}
\put(130,70){\vector(1,-1){68}}
\put(130,35){\vector(1,1){68}}
\put(130,0){\vector(1,1){68}}

%========== 4th line
\put(0,145){\small{$x(3)$}}
\put(0,140){\vector(1,0){28}}
\put(30,140){\circle*{5}}
\put(30,140){\vector(1,0){68}}
\put(100,140){\circle*{5}}
\put(100,145){\small{+}}
\put(100,140){\vector(1,0){28}}
\put(100,132){\small{+}}
%\put(110,145){\small{$W_{8}^{3}$}}
\put(130,140){\circle*{5}}
\put(130,140){\vector(1,0){68}}
\put(200,145){\small{+}}
\put(200,140){\circle*{5}}
\put(201,132){--}
\put(210,145){\small{$W_{8}^{2}$}}
\put(200,140){\vector(1,0){28}}
\put(230,140){\circle*{5}}
\put(230,140){\vector(1,0){68}}
\put(300,145){\small{+}}
\put(300,140){\circle*{5}}
\put(301,132){--}
\put(310,145){\small{$W_{8}^{0}$}}
\put(300,140){\vector(1,0){28}}
\put(330,140){\circle*{5}}
\put(330,140){\vector(1,0){38}}
\put(340,145){\small{$X(6)$}}
%========== 3rd line
\put(0,180){\small{$x(2)$}}
\put(0,175){\vector(1,0){28}}
\put(30,175){\circle*{5}}
\put(30,175){\vector(1,0){68}}
\put(100,175){\circle*{5}}
\put(100,180){\small{+}}
\put(100,175){\vector(1,0){28}}
\put(100,167){\small{+}}
%\put(110,180){\small{$W_{8}^{0}$}}
\put(130,175){\circle*{5}}
\put(130,175){\vector(1,0){68}}
\put(200,180){\small{+}}
\put(200,175){\circle*{5}}
\put(201,167){--}
\put(210,180){\small{$W_{8}^{0}$}}
\put(200,175){\vector(1,0){28}}
\put(230,175){\circle*{5}}
\put(230,175){\vector(1,0){68}}
\put(300,180){\small{+}}
\put(300,175){\circle*{5}}
\put(300,167){\small{+}}
%\put(310,25){\small{$W_{8}^{0}$}}
\put(300,175){\vector(1,0){28}}
\put(330,175){\circle*{5}}
\put(330,175){\vector(1,0){38}}
\put(340,180){\small{$X(2)$}}
\put(230,175){\vector(2,-1){68}}
\put(230,140){\vector(2,1){68}}
%========== 2nd line
\put(0,215){\small{$x(1)$}}
\put(0,210){\vector(1,0){28}}
\put(30,210){\circle*{5}}
\put(30,210){\vector(1,0){68}}
\put(100,210){\circle*{5}}
\put(100,215){\small{+}}
\put(100,210){\vector(1,0){28}}
\put(100,202){\small{+}}
%\put(110,215){\small{$W_{8}^{3}$}}
\put(130,210){\circle*{5}}
\put(130,210){\vector(1,0){68}}
\put(200,215){\small{+}}
\put(200,210){\circle*{5}}
\put(200,202){\small{+}}
%\put(210,215){\small{$W_{8}^{2}$}}
\put(200,210){\vector(1,0){28}}
\put(230,210){\circle*{5}}
\put(230,210){\vector(1,0){68}}
\put(300,215){\small{+}}
\put(300,210){\circle*{5}}
\put(301,202){--}
\put(310,215){\small{$W_{8}^{0}$}}
\put(300,210){\vector(1,0){28}}
\put(330,210){\circle*{5}}
\put(330,210){\vector(1,0){38}}
\put(340,215){\small{$X(4)$}}
%========== 1st line
\put(0,250){\small{$x(0)$}}
\put(0,245){\vector(1,0){28}}
\put(30,245){\circle*{5}}
\put(30,245){\vector(1,0){68}}
\put(100,245){\circle*{5}}
\put(100,250){\small{+}}
\put(100,245){\vector(1,0){28}}
\put(100,237){\small{+}}
%\put(110,250){\small{$W_{8}^{2}$}}
\put(130,245){\circle*{5}}
\put(130,245){\vector(1,0){68}}
\put(200,250){\small{+}}
\put(200,245){\circle*{5}}
\put(200,237){\small{+}}
%\put(210,250){\small{$W_{8}^{0}$}}
\put(200,245){\vector(1,0){28}}
\put(230,245){\circle*{5}}
\put(230,245){\vector(1,0){68}}
\put(300,250){\small{+}}
\put(300,245){\circle*{5}}
\put(300,237){\small{+}}
%\put(310,25){\small{$W_{8}^{0}$}}
\put(300,245){\vector(1,0){28}}
\put(330,245){\circle*{5}}
\put(330,245){\vector(1,0){38}}
\put(340,250){\small{$X(0)$}}

\put(230,245){\vector(2,-1){68}}
\put(230,210){\vector(2,1){68}}

\put(130,245){\vector(1,-1){68}}
\put(130,210){\vector(1,-1){68}}
\put(130,175){\vector(1,1){68}}
\put(130,140){\vector(1,1){68}}

\put(30,245){\vector(1,-2){69}}
\put(30,210){\vector(1,-2){69}}
\put(30,175){\vector(1,-2){69}}
\put(30,140){\vector(1,-2){69}}

\put(30,0){\vector(1,2){69}}
\put(30,35){\vector(1,2){69}}
\put(30,70){\vector(1,2){69}}
\put(30,105){\vector(1,2){69}}

\end{picture}
\vspace{3pt}\\
{\footnotesize Flow-graph of the decimation-in-frequency decomposition of an eight-point DFT}
\end{block}
\end{frame}

%%%%%%%% SLIDE 13
\begin{frame}
%\section{Signal Analysis with FFT}
\frametitle{Signal Analysis with FFT}

\begin{block}{Sample Signal} 
\begin{displaymath}
x(n) = \cos \left(2\pi\frac{200n}{N}\right) +0.5\sin \left( 2\pi\frac{50n}{N} \right)
\end{displaymath}
corrupted with zero-mean random noise
\end{block}

\begin{columns}
\column{0.5\textwidth}
\includegraphics[width=\textwidth]{x}
\column{0.5\textwidth}
\includegraphics[width=\textwidth]{ya}
\end{columns}

\end{frame}

%%%%%%%% SLIDE 14
\begin{frame}
%\section{Discrete Wavelet Transform (DWT)}
\frametitle{Discrete Wavelet Transform (DWT)}

%\subsection{Haar Transform}
\begin{block}{Haar Transform}
    \begin{itemize}
    \item Decomposes a discrete-time signal ($\mathbf{f}$) of length $N$ into two \textbf{sub-signals} of half its length: \\Trend ($\mathbf{a}$) –- running average and Fluctuation ($\mathbf{d}$) -– running difference
    \begin{displaymath}
    a_{m} = \frac{f_{2m}+f_{2m+1}}{\sqrt{2}},\quad d_{m} = \frac{f_{2m} - f_{2m+1}}{\sqrt{2}}
    \end{displaymath}
    
    \pause
    
    \item Simple Example: $\mathbf{f} = (4,6,10,12,8,6,5,5)$
{\scriptsize
\begin{tabular}{rccccccccccccccc}
  &   &  & & & &  & & & &  & & & &  &   \\ 
$\mathbf{f}$: & 4 & & 6 & & 10 & & 12 & & 8 & & 6 & & 5 & & 5 \\ 
\pause
 &   & $\searrow$$\swarrow$ & & & & $\searrow$$\swarrow$ & & & & $\searrow$$\swarrow$ & & & & $\searrow$$\swarrow$ &  \\ 
  &   & 5 & & & & 11 & & & & 7 & & & & 5 &   \\ 
\pause
 &   & $\downarrow$ & & & & $\downarrow$ & & & & $\downarrow$ & & & & $\downarrow$ &  \\
$\mathbf{a}^{1}$: & & $5\sqrt{2}$ & & & & $11\sqrt{2}$ & & & & $7\sqrt{2}$ & & & & $5\sqrt{2}$ & \\  
\pause
 &  & &  & &  & &  & &  & &  & &  & &  \\ 
$\mathbf{f}$: & 4 & & 6 & & 10 & & 12 & & 8 & & 6 & & 5 & & 5 \\ 
\pause
 &   & $\searrow$$\swarrow$ & & & & $\searrow$$\swarrow$ & & & & $\searrow$$\swarrow$ & & & & $\searrow$$\swarrow$ &  \\
 &   & $-1$ & & & & $-1$ & & & & $1$ & & & & $0$ &   \\ 
\pause
 &   & $\downarrow$ & & & & $\downarrow$ & & & & $\downarrow$ & & & & $\downarrow$ &  \\
$\mathbf{d}^{1}$: & & $-\sqrt{2}$ & & & & $-\sqrt{2}$ & & & & $\sqrt{2}$ & & & & $0$ &   
\end{tabular}
}
\pause
	\item Extended to multiple levels
	\end{itemize}
\end{block}
\end{frame}

%%%%%%%%%%%%%
\begin{frame}
\frametitle{Discrete Wavelet Transform (DWT) \hfill{2}}
\framesubtitle{Haar Transform}
\begin{block}{Haar Wavelets ($\mathbf{W}$) and Scaling Functions ($\mathbf{V}$)}
            \begin{itemize}
            \item Basis set of elementary signals to express Haar transform as scalar products
            \item First level: $d_{m} = \mathbf{f} \cdot \mathbf{W}_{m}^{1}, \qquad a_{m} = \mathbf{f} \cdot \mathbf{V}_{m}^{1}$
            \end{itemize}
\end{block}

\pause

\begin{columns}
\column{0.5\textwidth}
\begin{block}{Wavelets}
{\scriptsize
\begin{eqnarray*}
\mathbf{W}_{0}^{1} & = &\left( \frac{1}{\sqrt{2}},\frac{-1}{\sqrt{2}},0,0,0,0,\ldots,0 \right) \\
\mathbf{W}_{1}^{1} & = & \left( 0,0,\frac{1}{\sqrt{2}},\frac{-1}{\sqrt{2}},0,0,\ldots,0 \right) \\
	 & \vdots &  \\
\mathbf{W}_{N/2-1}^{1} & = & \left( 0,0,0,0,\ldots,0,\frac{1}{\sqrt{2}},\frac{-1}{\sqrt{2}} \right) 
\end{eqnarray*}
}
\end{block}

\pause

\column{0.5\textwidth}
\begin{block}{Scaling Signals}
{\scriptsize
\begin{eqnarray*}
\mathbf{V}_{0}^{1} & = &\left( \frac{1}{\sqrt{2}},\frac{1}{\sqrt{2}},0,0,0,0,\ldots,0 \right) \\
\mathbf{V}_{1}^{1} & = & \left( 0,0,\frac{1}{\sqrt{2}},\frac{1}{\sqrt{2}},0,0,\ldots,0 \right) \\
	 & \vdots & \\
\mathbf{V}_{N/2-1}^{1} & = & \left( 0,0,0,0,\ldots,0,\frac{1}{\sqrt{2}},\frac{1}{\sqrt{2}} \right) 
\end{eqnarray*}
}
\end{block}
\end{columns}
\end{frame}

%%%%%%%%%%%%%%%%%%% Slide
\begin{frame}
\frametitle{Discrete Wavelet Transform (DWT) \hfill{3}}
\framesubtitle{Haar Transform}
\begin{block}{Reconstruction and Multi-resolution Analysis (MRA)}
            \begin{itemize}
            \item First level: $\mathbf{f}=\mathbf{A}^{1}+\mathbf{D}^{1}$
            \begin{eqnarray*}
            \mathbf{A}^{1} &=& a_{0}\mathbf{V}_{0}^{1}+a_{1}\mathbf{V}_{1}^{1}+\ldots+a_{N/2-1}\mathbf{V}_{N/2-1}^{1} \\
                    \mathbf{D}^{1} &=& d_{0}\mathbf{W}_{0}^{1}+d_{1}\mathbf{W}_{1}^{1}+\ldots+d_{N/2-1}\mathbf{W}_{N/2-1}^{1}
	\end{eqnarray*}
	    
	\pause
	
    	\item Example:\\
{\footnotesize
    \begin{tabular}{rrrrrrrrr}
	&  &  &  &  &  &  & & \\
	$\mathbf{f}$: & $4$ & $6$ & $10$ & $12$ & $8$ & $6$ & $5$ & $5$ \\
	&  &  &  &  &  &  & & \\
  	\pause
    	$\mathbf{A}^{1}$: & $5$ & $5$ & $11$ & $11$ & $7$ & $7$ & $5$ & $5$ \\
    \pause
    $\mathbf{D}^{1}$: & $-1$ & $1$ & $-1$ & $1$ & $1$ & $-1$ & $0$ & $0$
    \end{tabular}
}
	    \pause
            \item Multiple levels:
            \end{itemize}
\begin{picture}(150,45)(-100,0)
\setlength{\unitlength}{0.5pt}
{\scriptsize
%\begin{picture}(300,320)(0,0)
\put(0,0){\framebox(25,20){$\mathbf{A}^{3}$}}
\put(60,0){\framebox(25,20){$\mathbf{D}^{3}$}}
\put(30,30){\framebox(25,20){$\mathbf{A}^{2}$}}
\put(90,30){\framebox(25,20){$\mathbf{D}^{2}$}}
\put(60,60){\framebox(25,20){$\mathbf{A}^{1}$}}
\put(120,60){\framebox(25,20){$\mathbf{D}^{1}$}}
\put(90,90){\framebox(25,20){$\mathbf{f}$}}

\put(12,40){\line(1,0){18}}
\put(12,40){\vector(0,-1){20}}
\put(55,40){\line(1,0){18}}
\put(73,40){\vector(0,-1){20}}

\put(42,70){\line(1,0){18}}
\put(42,70){\vector(0,-1){20}}
\put(85,70){\line(1,0){18}}
\put(103,70){\vector(0,-1){20}}

\put(72,100){\line(1,0){18}}
\put(72,100){\vector(0,-1){20}}
\put(115,100){\line(1,0){18}}
\put(133,100){\vector(0,-1){20}}
%\end{picture}

}
\end{picture}
\end{block}
\end{frame}

%%%%%%%% SLIDE 12
\begin{frame}
\frametitle{Discrete Wavelet Transform (DWT)\hfill{4}}

%\subsection{Daubechies Transform}
\begin{block}{Daubechies Transform}
    \begin{itemize}
    \item \alert{Family} of Daubechies Wavelets (dbn)
    \item $\mathbf{db1}$ is \textbf{Haar}
    \item $\mathbf{db2}$
            \begin{itemize}
            \item First Level Scaling Numbers:
            \begin{displaymath}
            \alpha_{0} = \frac{1+\sqrt{3}}{4\sqrt{2}}, \quad \alpha_{1}=\frac{3+\sqrt{3}}{4\sqrt{2}}, \quad \alpha_{2} = \frac{3-\sqrt{3}}{4\sqrt{2}}, \quad \alpha_{3}=\frac{1-\sqrt{3}}{4\sqrt{2}}
            \end{displaymath}

            \item First Level Wavelet Numbers:
            \begin{displaymath}
            \beta_{0} = \frac{1-\sqrt{3}}{4\sqrt{2}}, \quad \beta_{1}=\frac{\sqrt{3}-3}{4\sqrt{2}}, \quad \beta_{2} = \frac{3+\sqrt{3}}{4\sqrt{2}}, \quad \beta_{3}=\frac{-1-\sqrt{3}}{4\sqrt{2}}
            \end{displaymath}

	    \end{itemize}
    \end{itemize}
\end{block}
\end{frame}

%%%%%%%%%%%%%%%%%%%%%%%%
\begin{frame}
\frametitle{Discrete Wavelet Transform (DWT)\hfill{5}}
\framesubtitle{Daubechies Transform}

\begin{columns}
\column{0.5\textwidth}
\begin{block}{\textbf{db2} Scaling Signals}
{\scriptsize
\begin{eqnarray*}
\mathbf{V}_{0}^{1} & = & ( \alpha_{0},\alpha_{1},\alpha_{2},\alpha_{3},0,0,\ldots,0 ) \nonumber \\
\mathbf{V}_{1}^{1} & = & ( 0,0,\alpha_{0},\alpha_{1},\alpha_{2},\alpha_{3},\ldots,0 ) \nonumber \\
	 & \vdots & 	\nonumber \\
\mathbf{V}_{N/2-2}^{1} & = & ( 0,0,\ldots,0,\alpha_{0},\alpha_{1},\alpha_{2},\alpha_{3} ) \nonumber \\ 
\mathbf{V}_{N/2-1}^{1} & = & ( \alpha_{2},\alpha_{3},0,0,\ldots,0,\alpha_{0},\alpha_{1} )  
\end{eqnarray*}
}
\end{block}
\column{0.5\textwidth}
\begin{block}{\textbf{db2} Wavelets}
{\scriptsize
\begin{eqnarray*}
\mathbf{W}_{0}^{1} & = & ( \beta_{0},\beta_{1},\beta_{2},\beta_{3},0,0,\ldots,0 ) \nonumber \\
\mathbf{W}_{1}^{1} & = & ( 0,0,\beta_{0},\beta_{1},\beta_{2},\beta_{3},\ldots,0 ) \nonumber \\
	 & \vdots & 	\nonumber \\
\mathbf{W}_{N/2-2}^{1} & = & ( 0,0,\ldots,0,\beta_{0},\beta_{1},\beta_{2},\beta_{3} ) \nonumber \\ 
\mathbf{W}_{N/2-1}^{1} & = & ( \beta_{2},\beta_{3},0,0,\ldots,0,\beta_{0},\beta_{1} )  
\end{eqnarray*}
}
\end{block}
\end{columns}
\end{frame}

%%%%%%%% SLIDE 13
\begin{frame}
%\section{Signal Analysis with DWT}
\frametitle{Signal Analysis with DWT}
\framesubtitle{Haar and Daubechies Transforms}
\begin{block}{Example 1: Trend and fluctuation subsignals (concatenated)}
Signal: $g(t)=20t^{2}(1-t)^{4}\cos(12\pi t)$ over the interval $[0,1)$. 
\end{block}
 
\begin{columns}
\column{0.5\textwidth}
\includegraphics[width=\textwidth]{example}
\column{0.5\textwidth}
\includegraphics[width=\textwidth]{exampledb}
\end{columns}
\end{frame}

%%%%%%%% SLIDE 14
\begin{frame}
\frametitle{Signal Analysis with DWT \hfill{2}}
\framesubtitle{Haar and Daubechies Decomposition}

\begin{block}{Example 2} 
{\scriptsize Signal: Simple step}
\end{block}
%\begin{columns}
%\column{0.5\textwidth}
\begin{center}
\includegraphics[width=0.75\textwidth]{step}
\end{center}
%\column{0.5\textwidth}
%\includegraphics[width=\textwidth]{exampledb}
%\end{columns}
\end{frame}

%%%%%%%% SLIDE 15
\begin{frame}
\frametitle{Signal Analysis with DWT \hfill{3}}
\framesubtitle{Haar and Daubechies Decomposition}

\begin{block}{Example 3} 
{\scriptsize Signal: ``frequency breakdown''}
\end{block}
\begin{center}
\includegraphics[width=0.75\textwidth]{frbd}
\end{center}

\end{frame}

%%%%%%%% SLIDE 16
\begin{frame}
\frametitle{Signal Analysis with DWT \hfill{4}}
\framesubtitle{Haar and Daubechies Decomposition}

\begin{block}{Example 4}
{\scriptsize
    Signal: $x(n)$ formed with $N=1024$ samples \\
    $x(n)=\sin(2\pi\frac{5n}{N})$ from $n=0$ to $n=307$ and from $n=716$ to $n=1023$ \\
    $x(n)=1.25\sin(2\pi\frac{15n}{N})$ from $n=308$ to $n=715$
}
\end{block}

\begin{center}
\includegraphics[width=0.75\textwidth]{mysignal}
\end{center}
\end{frame}

%%%%%%%% SLIDE 17
\begin{frame}
%\section{Summary and Conclusion}
\frametitle{Summary and Conclusion}

\begin{block}{Condition Monitoring}
\begin{itemize}
\item A very important technology in the field of electrical equipment maintenance
\item Now a topic of immense interest to power system engineers as well as researchers

\item Advanced signal processing techniques are of huge prospect in developing novel condition monitoring systems
\end{itemize}
\end{block} 
\begin{block}{Approaches} Development of computer programs for signal processing using FFT and DWT algorithms \end{block}
\begin{block}{Future Plan}
\alert{Machinery Fault Simulator (MFS)} by SpectraQuest, Inc., USA to simulate various motor faults.
\end{block} 
\end{frame}

%%%%%%%% SLIDE 18
\begin{frame}
%\section{References}
\frametitle{References}

\begin{itemize} 
    \item J. Penman, M.N. Dey, A.J. Tait and W.E. Bryan, ``Condition Monitoring of Electrical Devices'', IEE Proc., \textbf{133}, Part B(3):164-180, 1986.
    \item P.J. Tavner, B.G. Gaydon and D.M. Word, ``Monitoring Generators and Large Motors'', IEE Proc., \textbf{133}, Part B(3):181-189, 1986
    \item P.J. Tavner and J. Penman, \emph{Condition Monitoring of Electrical Machines}, Research Studies Press, Ltd., 1987.
    \item M.E.H. Benbouzid, ``A Review of Induction Motors Signature Analysis as a Medium  for  Faults  Detection'',  IEEE  Transactions on Industrial Electronics, \textbf{47}(5):984-993, October 2000.
    \item A. Graps, ``An Introduction to Wavelets'', IEEE Computational Science and Engineering, \textbf{2}(2):50-61, Summer 1995.
    \item Steven C. Chapra and Raymond P. Canale, \emph{Numerical Methods for Engineers with Programming and Software application}, Tata McGraw-Hill Publishing Co. Ltd., 3rd edition, 2000
    \item James S. Walker, \emph{A Primer on Wavelets and their Scientific Applications}, Chapman \& Hall/CRC, 1999.
\end{itemize}
\end{frame}

\begin{frame}
\begin{center}
\LARGE{Thank you}
\end{center}
\end{frame}

\end{document}
