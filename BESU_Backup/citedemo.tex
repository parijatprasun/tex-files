\documentclass[11pt]{article}
% LaTeX document by David J C MacKay
% AIMS November 2004
%
% We recommend that you handle all your citations 
% and your bibliography using BibTeX.
%
% This is an example document illustrating 
% how to use BibTeX to make your bibliography.
%
% PACKAGES
%
\usepackage{epsfig}%    A package for including postscript figures
\usepackage{fancybox}%  Provides ability to put verbatim text inside boxes
\usepackage{natbib}%%%%  To change the citation style, modify this line
% \usepackage{noindent}% This changes paragraph style
%
% PAGE SIZE
%
\textheight=10.25in
\textwidth=6.7in
\oddsidemargin=-0.245in
\topmargin=-1.0in
%
% Pick a bibliography-style. I recommend abbrvnat. 
%  The natbib package supports many styles.
%  Change this style to change the appearance of the bibliography.
%
%\bibliographystyle{unsrtnat}
%\bibliographystyle{plainnat}
\bibliographystyle{abbrvnat}
%
% DOCUMENT CONTENTS
%
\begin{document}
\title{How to make your bibliography with BibTeX}
\author{David J. C. MacKay\\
 AIMS\\
 {\tt http://www.aims.ac.za/$\sim$mackay/tex/}
}
\date{\today\ -- Version 2.1}
\maketitle
 You {{can}} read the source of this document in 
 {\tt{/home/mackay/tex/citedemo.tex}}.
 My {\tt{.bib}} file is
 {\tt{/home/mackay/bibs.bib}}.
 Feel free to use these files as templates.

\section{Overview}
 BibTeX creates your bibliography from information 
 in  a {\tt{.bib}} file, which describes articles 
 and books in a general format (see figure \ref{example1}a).
 You can write the entries 
 in your {\tt{.bib}} file yourself, or copy them from 
 other people.  (BibTeX is widely used, and the
 internet contains
  {\tt{.bib}} entries for most articles and books.)
 If you edit your {\tt{.bib}} file using {\tt{emacs}}, 
 you will find the {\tt{emacs}} toolbar offers lots of helpful 
 operations for adding and manipulating entries.  
 BibTeX reads your 
  {\tt{.bib}} file  and makes a bibliography  entry
 for each 
 article you cite in your {\tt{.tex}} file. 
 BibTex puts these  bibliography  entries in  a {\tt{.bbl}}
 file, which \LaTeX\ then includes in your 
 document.  
 BibTeX automatically  produces a bibliography 
 with a consistent style.
 If you want to change the style of the bibliography, 
 you need only change one line in your {\tt{.tex}} file. The 
 next time you run BibTeX, it  will reread  the  {\tt{.bib}} file
 and reformat the bibliography accordingly. 
 

\begin{figure}[hbtp] 
\begin{center}
\footnotesize
\begin{tabular}{cc}
\begin{tabular}[b]{p{3.4in}}
%\footnotesize
 ~\\[-0.2in]
\begin{verbatim}
@article{Shannon48,
 author = {Shannon, C. E.},
 title = {A Mathematical Theory of Communication},
 journal="Bell Sys. Tech. J.",
 volume = 27,
 pages ="379-423, 623-656",
 year = 1948
}
\end{verbatim}
\end{tabular}
&
\begin{tabular}[t]{c} ~\\[0.2in]
\epsfig{figure=/home/mackay/tex/figs/balance.eps,width=2in}
\end{tabular}
\\[-0.2in]
(a) & (b) \\[-0.2in]
\end{tabular}
\end{center}
\caption{(a) An example {\tt{.bib}} entry.
         Each field (author, title, journal, \ldots)
 is surrounded either  by  braces {\tt\{}\ldots{\tt{\}}}
 or by quotes {\tt"}\ldots{\tt"}.
  (b)   The 12-ball weighing problem, illustrating 
 the use of {\tt{epsfig}}.}
\label{example1}
\end{figure}

\subsection{Citation styles}
 Citations in articles and books come in several forms. 
 Some journals require you to use {\em numerical\/} citations:
\setlength{\fboxsep}{10pt}
\begin{center}
\begin{Sbox}
\begin{minipage}{5.5in}
%\begin{quote}
  Good error-correcting codes exist  [13]. 
\medskip

 Shannon [13] proved that reliable communication is 
 possible.
%\end{quote}
\end{minipage}
\end{Sbox}
\ovalbox{\TheSbox}
\end{center}
%
 Others prefer  an {\em author--year\/} style:
\begin{center}
\begin{Sbox}
\begin{minipage}{5.5in}
%\begin{quote}
  Good error-correcting codes exist  (Shannon, 1948). 
\medskip

 Shannon (1948) proved that reliable communication is 
 possible.
%\end{quote}
\end{minipage}
\end{Sbox}
\ovalbox{\TheSbox}
\end{center}

 We recommend using an author--year style wherever possible
 because  it is   more reader-friendly. 

 We recommend using the {\tt{natbib}} package
 because 
 it is compatible  with both  citation styles.
 If you write a  paper and decide to change your citation style
 from author--year to numerical, 
 you need to change only one line in your {\tt{.tex}} file;
  all the citations will be changed automatically.
%  from author--year style  to numerical 

\section{Using {\tt{natbib}}  in your {\tt{.tex}} file}
\subsection{Starting and finishing}
 Your  {\tt{.tex}}  file should have
 the lines 
 {\verb+\usepackage{natbib}+} 
 and 
 {\verb+\bibliographystyle{abbrvnat}+} 
 before  
 {\verb+\begin{document}+}.

 At the end, put the command {\verb+\bibliography{your_bib_file}+}
 where you want the bibliography
 to appear. In my file, for example,  I use the command 
 {\verb+\bibliography{/home/mackay/bibs.bib}+}.

\subsection{How to cite}
 There are two types of  citation command: 
 {\verb+\citet+} for {\em textual\/}
 and
  {\verb+\citep+} for
  {\em parenthetical\/} citations. 
%  the two types of citation  work like this.
 The two sentences
\begin{quote}
  Good error-correcting codes exist \citep{Shannon48}.

  \citet{Shannon48} proved that reliable communication is possible. 
\end{quote}
 are produced by the following \LaTeX:
\begin{quote}
\tt
  Good error-correcting codes exist \verb+\citep{Shannon48}+.

  \verb+\citet{Shannon48}+ proved that reliable communication is possible. 
\end{quote}
 The string {\tt{Shannon48}} is the  {\em{key}\/} used to identify the 
 corresponding {\tt{.bib}} entry in my {\tt{bibs.bib}} file, 
 which was shown in figure \ref{example1}a.
 This key works in the same way as the {\em labels\/} that you use
 to refer to equations and figures. 

 Textual citation is sometimes called `citing as a noun'. 
 Other citation commands, citation styles, and package options  
 are described in the {\tt{natbib}}
 documentation, which you can find by 
 typing {\tt{locate natbib}} in an xterm, or
 by  searching on Google.
 You can read the {\tt{natbib}} manual with this command: 
\begin{quote}
 \tt xdvi /usr/share/doc/texmf/latex/natbib/natbib.dvi.gz
\end{quote}

\section{How to run BibTeX}
 Normally when you use \LaTeX, you have to run 
\begin{quote}
 \tt latex file
\end{quote}
 a couple of times, where {\tt file.tex} is your {\tt{.tex}} file. 
 Now, you need to run {\tt{bibtex}} too. The normal sequence is:
\begin{quote}
 \tt latex file\\
 \tt bibtex file \ \ \ \ 
   {\em - give the name of your {\tt{.tex}} file here.} \\
 \tt latex file\\
 \tt latex file
\end{quote}
 I use a makefile ({\tt{/home/mackay/tex/Makefile}}) 
 to run  BibTeX and \LaTeX\ at the appropriate times.
 After {\tt{bibtex}} runs, and after 
   {\tt{latex}} runs  for the third time, 
 see if there are any error messages. 
 The most common 
 causes of errors are: incorrectly formatted 
 {\tt{.bib}} entries in the {\tt{.bib}} file,
 and incorrect {\verb+\cite+} commands in the 
 {\tt{.tex}} file. 
 Sometimes when you  fix an error, \LaTeX\  
 remains confused. If so, give  \LaTeX\ 
 a fresh start  by removing the {\tt{.bbl}}
 and {\tt{.aux}} files created by 
 BibTeX and \LaTeX.

\bibliography{bibs.bib}


\end{document}

\begin{center}
{\tt http://www.aims.ac.za/$\sim$mackay/tex/}
\end{center}


%\citet{MacKay:itp}

\section{Data Compression}
 Why is $h({\bf x}) \equiv \log_2  1/P({\bf x})$ a good measure of the 
 information content of the outcome $\bf x$?
 

